\section{Algebraic structures}
\label{sec:Algebraic-structures}

An algebraic 
structure~\cite{
wiki:Algebraic-structure,
wiki:Mathematical-structure,
wiki:Outline-of-algebraic-structures}
consists of:
\begin{itemize}
  \item A primary set --- the \textit{elements} of the structure.
  \item zero or a few auxilliary sets.
  \item functions (called \textit{operations})
that take a small number of arguments from one or more of the sets
and return elements of the sets.
\end{itemize}

The type of an algebraic structure corresponds to identities
that the operations satisfy.
(\textbf{TODO:} Examples of operation identities?)

Unfortunately, the names for algebraic structures 
are, as a rule, not very informative.


Other structures:
\begin{itemize}
  \item Order
  \item Measure
  \item Metric
  \item Geometry
  \item Topology
\end{itemize}

For \glssymbol{RealNumbers}:
Order and Metric induce Topology.
Order and Algebraic structure lead to ordered field.
Algebraic structure and topology make Lie group.

Some relevant examples:

\subsection{Monoid}

Set $\Set{S}$ and operation $\diamond$ such that,
if $a, b, c \in \Set{S}$, then
\begin{description}
\item[Closed] $a \diamond b = \diamond \left( a, b \right) 
= \left( \diamond \; a \; b \right) \in \Set{S}$
\item[Associative] $a \diamond b \diamond c =
 \left( a \diamond b \right) \diamond c =  
 a \diamond \left( b \diamond c \right) $
 \item[Identity] There is an $i \in \Set{S}$ such that 
 $i \diamond a = a \; \forall a \in \Set{S}$.
 Exercise: show that $i$ is unique.
 \end{description}

Example: 
$\Set{S}$ the functions from some domain $\Set{X}$ to itself. 
Operation $\diamond$ is function composition:
$\left( f \diamond g \right) (x) = 
f \left( g \left( x \right) \right)$.

\subsection{Group}

Monoid $\left[ \Set{S}, \diamond \right]$ such that
\begin{description}
 \item[Inverse] For every $a \in \Set{S}$ there exists an
 $a^{-1}$ such that $a^{-1} \diamond a = i$.
 Exercise: show that this implies that $a \diamond a^{-1} = i$ .
\end{description}

\subsection{Commutative group}

(aka abelian group.)

A group $\left[ \Set{S}, \diamond \right]$ where
\begin{description}
 \item[Commutative] $a \diamond b = b \diamond a \; \forall a,b \in \Set{S}$.
\end{description}

Example: 
$\left[ \glssymbol{Integers}, *_{\glssymbol{Integers}} \right]$

\subsection{Rings}
\label{sec:Rings}
\cite{wiki:Ring-mathematics}

A set and 2 operations: $\left[ \Set{S}, +, * \right]$
where
\begin{description}
  \item[Addition group] $\left[ \Set{S}, + \right]$ 
  is a commutative group (with the identity written $0$).
  \item[Multiplication monoid] $\left[ \Set{S}, * \right]$ 
  is a monoid (with the identity written $1$).
  Note this means $*$ may not commute. If it does, 
  then this is a \textit{commutative ring}.
  \item[Distributive] $a * \left( b + c \right) 
  = \left( a * b \right) + \left( a * c \right)$
\end{description}

Example:

\subsection{Fields}
\label{sec:Fields}
\cite{wiki:Field-mathematics}

A ring $\left[ \Set{S}, +, * \right]$ where
$*$ is commutative and the nonzero elements of $\Set{S}$
have multiplicative inverses.

Without refering to other structures:
A \textit{field} is set and 2 operations: 
$\left[ \Set{S}, +, * \right]$
where
\begin{description}
  \item[Additive closure] $a + b \in \Set{S} \; 
  \forall \, a,b \in \Set{S}$ 
  \item[Additive associativity] 
  $\left( a + b \right) + c = a + \left( b + c \right)$ 
  \item[Additive commutativity] $a + b = b + a$ 
  \item[Additive identity] $\exists \, 0 \in \Set{S} \text{ s.t. } 
  a + 0 = 0 + a = a \; \forall \, a \in \Set{S}$ 
  \item[Additive inverse] $\forall a \in \Set{S} \; 
  \exists -a \in \Set{S} 
  \text{ s.t. }  a + \left( -a \right) = \left( -a \right) + a = 0$ 
  \item[Multiplicative closure] $a * b \in \Set{S} \; \forall \,a,b \in \Set{S}$ 
  \item[Multiplicative associativity] 
  $\left( a * b \right) * c = a * \left( b * c \right)$ 
  \item[Multiplicative commutativity] $a * b = b * a$ 
  \item[Multiplicative identity] $\exists 1 \in \Set{S}
   \text{ s.t. } 
  a * 1 = 1 * a = a \; \forall \, a \in \Set{S}$ 
  \item[Multiplicative inverse] $\forall a \ne 0 \in \Set{S} 
  \exists a^{-1} \in \Set{S}
  \text{ s.t. }  a * a^{-1} = a^{-1} * a = 1$ 
  (Note the restriction to elements other than the additive identity.) 
  \item[Distributive] $a * \left( b + c \right) 
  = \left( a * b \right) + \left( a * c \right)$
\end{description}


Examples: 
\glssymbol{RationalNumbers}, \glssymbol{RealNumbers}.

\subsection{Floating point numbers as an algebraic structure}

Two non-associative commutative operations on a finite,
almost ordered set.

\textbf{TODO:} do TwoPlus and TwoMutliply 
suggest any way to get associative operations?
Idea is to carry error accumulators so 
any sequence of additions and multiplies 
returns the correct \glssymbol{RealNumbers} value rounded to float.
EG: $ a +_{\text{fl}} b = c +_{\text{fl}} \epsilon$
where $\text{fl} c +_{\glssymbol{RealNumbers}} \epsilon) = c$
Absorbing elements at $\pm\infty$, \textt{NaN} make that hard.
Is it possible to replace $\pm\infty$ and \textt{NaN}
by symbolic expressions that can later be combined to get 
accurate finite values?



