\chapter{Interpolation}\label{ch:Interpolation}

Motivation: 
\begin{enumerate}
 \item Minimization of real-valued functions on linear spaces.
 \item Iterated line search methods.
 \item $1$d minimization of real-valued functions on a line,
 identified with \glssymbol{RealNumbers}.
 \item $1$d minimization methods based on visiting the
 argmin of a model function.
 \item model function constructed using values and derivatives
 of the $1$d objective function at recently visited locations.
 \item common model function are low degree polynomials 
 interpolating the supplied values and derivatives.
\end{enumerate}

$1$d minimization methods only use the argmin of the model 
function, so constructing a representation of the interpolating 
polynomial is usually unnecessary expense.
This code is intended to make it easy to either 
compute the argmin alone,
or reify the model function, for visualization and debugging.
(Maybe a lazy polynomial constructor that provides argmin with
minimum effort and defers other work until first call to 
\texttt{value} or \texttt{derivative}.)

The polynomials of degree $k$ for a linear space of dimension
$k+1$.
The most common interpolating polynomials are quadratic or cubic,
so we will have $3$ or $4$ dimensions (degrees of freedom) to play
with. 
(I will discuss affine and constant interpolation for 
completeness.)

Why not higher order? Are other interpolating functions useful?
What about splines?

An interpolating polynomial matches some number of $x,y$
and $x,d$ pairs, where $x$ is a location in the domain,
$y$ a value in the codomain (range) and $d$ the slope.

The polynomials of degree $k$ for a linear space of dimension
$k+1$. 

A degree $k$ polynomial can match a combination of $k+1$ value 
and/or 
slope constraints, where there are at most $k$ slope constraints,
and where the value constraints are at distinct $x$'s and the
slope constraint are at distinct $x$'s, but a slope and a value 
constraint can be given at the same $x$.

A key issue is the choice of basis.

Some bases are may be more convenient when constructing a 
polynomial by interpolation.

Another consideration is the accuracy and cost of 
argmin \texttt{argmin}, \texttt{value}, and \texttt{derivative} 
when the interpolating polynomial is approximated with 
floating point numbers.

\texttt{BigFraction} implementation

Accuracy of evaluating the interpolating polynomial
versus how far the interpolating function is from the function it
interpolates.

Extrapolation often results when using the \texttt{argmin}.
Relationship between \texttt{argmin} and inverse interpolation.

\section{Monomial basis}\label{sec:monomial-basis}

The most common basis is the monomial:
$p(x) = a_0 + a_1 x + a_2 x^2 + a_3 x^3$.
$p$ is quadratic if $a_3$ is $0$,
affine if $a_3$ and $a_2$ are $0$,
and constant when only $a_0$ is not $0$.
The basis functions are:
\begin{enumerate}[start=0]
 \item $p_0(x) = 1$
 \item $p_1(x) = x$
 \item $p_2(x) = x^2$
 \item $p_3(x) = x^3$
\end{enumerate}

(Note: these basis function aren't orthogonal or normalized)is any
sense. In fact, choosing a distance or an inner product for
a polynomial space is not a simple question.)

Constructing an interpolating polynomial in the monomial basis
requires solving a system of equations.

Suppose, for example, we have $x_0,y_0,d_0$ and
$x_1,y_1,d_1$, that is, we have specified the value and slope we
want at $x_0 \neq x_1$.
That's $4$ constraints, impying a cubic interpolant.
To determine the monomial basis coefficients,
solve the linear system:

\begin{equation}
\begin{pmatrix}
y_0 \\ d_0 \\ y_1 \\ d_1
\end{pmatrix}
=
\begin{pmatrix}
1 & x_0 & \phantom{2} x_0^2 & \phantom{3} x_0^3 \\
  & 1   & 2 x_0 & 3 x_0^2 \\
1 & x_1 & \phantom{2} x_1^2 & \phantom{3} x_1^3 \\
  & 1   & 2 x_1 & 3 x_1^2 
\end{pmatrix}
\begin{pmatrix}
a_0 \\ a_1 \\ a_2 \\ a_3
\end{pmatrix}
\end{equation}

\begin{align}
 y_0 & = a_0 + a_1 x_0 + \phantom{2} a_2 x_0^2 + \phantom{3} a_3 x_0^3 \\
 d_0 & = \phantom{a_0} \phantom{+} a_1 \phantom{x_0} + 2 a_2 x_0 + 3 a_3 x_0^2 \nonumber \\
 y_1 & = a_0 + a_1 x_1 + \phantom{2} a_2 x_1^2 + \phantom{3} a_3 x_1^3 \nonumber \\
 d_1 & = \phantom{a_0} \phantom{+} a_1 \phantom{x_0} + 2 a_2 x_1 + 3 a_3 x_1^2 \nonumber
\end{align}

\begin{align}
 y_0 = & {a_0 + a_1 x_0 + \phantom{2} a_2 x_0^2 + \phantom{3} a_3 x_0^3} \\
 d_0 = & \pushright{a_1 \phantom{x_0} + 2 a_2 x_0 + 3 a_3 x_0^2} \nonumber \\
 y_1 = & {a_0 + a_1 x_1 + \phantom{2} a_2 x_1^2 + \phantom{3} a_3 x_1^3 }\nonumber \\
 d_1 = & \pushright{a_1 \phantom{x_0} + 2 a_2 x_1 + 3 a_3 x_1^2} \nonumber
\end{align}

\begin{align}
 y_0 = & {a_0 + a_1 x_0 + a_2 x_0^2 + a_3 x_0^3} \\
 d_0 = & \pushright{a_1 + 2 a_2 x_0 + 3 a_3 x_0^2} \nonumber \\
 y_1 = & {a_0 + a_1 x_1 + a_2 x_1^2 + a_3 x_1^3 }\nonumber \\
 d_1 = & \pushright{a_1 + 2 a_2 x_1 + 3 a_3 x_1^2} \nonumber
\end{align}

\begin{align}\label{eq:hermite-eqns}
 y_0 & = a_0 + a_1 x_0 + a_2 x_0^2 + a_3 x_0^3 \\
 d_0 & = a_1 + 2 a_2 x_0 + 3 a_3 x_0^2 \nonumber \\
 y_1 & = a_0 + a_1 x_1 + a_2 x_1^2 + a_3 x_1^3 \nonumber \\
 d_1 & = a_1 + 2 a_2 x_1 + 3 a_3 x_1^2 \nonumber 
\end{align}

\begin{align}
 y_0 = &a_0 + a_1 x_0 + a_2 x_0^2 + a_3 x_0^3 \\
 d_0 = &\specialcell{\hfill a_1 + 2 a_2 x_0 + 3 a_3 x_0^2} \nonumber \\
 y_1 = &a_0 + a_1 x_1 + a_2 x_1^2 + a_3 x_1^3 \nonumber \\
 d_1 = &\specialcell{{\hfill a_1 + 2 a_2 x_1 + 3 a_3 x_1^2}} \nonumber 
\end{align}

\subsection{Constant Monomial}

The only possibility is to 'interpolate' $(x_0,y_0)$ with
$a_0 = y_0$ (and $a_i = 0$ for other $i$). 

\subsection{Affine Monomial}

There are $2$ possibilities, finding the line thru $2$ points,
$(x_0,y_0)$ and $(x_1,y_1)$, or or matching the value and slope at
$(x_0,y_0,d_0)$.

In the first case:
\begin{align}
  y_0 & = \mu_0+\mu_1 x_0  \\
   y_1 & = \mu_0+\mu_1 x_1  
\end{align}

 
\begin{align}
  \mu_0 & = \frac
{x_0 y_1 - x_1 y_0}
{x_0 - x_1} \\
   \mu_1 & = \frac
{y_0 - y_1}
{x_0 - x_1} 
\end{align}


In the second:
\begin{align}
  y_0 & = \mu_0+\mu_1 x_0  \\
   d_1 & = \mu_1  
\end{align}

 
\begin{align}
  \mu_0 & =  - d_1 x_0+y_0  \\
   \mu_1 & = d_1  
\end{align}


\subsection{Quadratic Monomial}

\subsubsection{$yyy$}

\begin{align}
  y_0 & = a_0+a_1 x_0+a_2 x_0^{2}  \\
   y_1 & = a_0+a_1 x_1+a_2 x_1^{2}  \\
   y_2 & = a_0+a_1 x_2+a_2 x_2^{2}  
\end{align}

 
\begin{align}
   a_0 & = \frac
{\left(x_1 y_2 - x_2 y_1\right) x_0^{2}+\left(x_1 - x_2\right) x_1 x_2 y_0 -  \left(x_1^{2} y_2 - x_2^{2} y_1\right) x_0}
{\left(x_0 - x_1\right)  \left(x_0 - x_2\right) \left(x_1 - x_2\right)} \\
   a_1 & = \frac
{\left(y_0 - y_1 \right) x_2^{2} - \left(y_0 - y_2\right) x_1^{2}+\left(y_1 - y_2\right) x_0^{2} }
{\left(x_0 - x_1\right) \left(x_0 - x_2\right) \left(x_1 - x_2\right) } \\
   a_2 & = \frac
{ - \left(\left(y_0 - y_1\right) x_2 - \left(y_0 - y_2\right) x_1+ \left(y_1 - y_2\right) x_0\right)}
{\left(x_0 - x_1\right) \left(x_0 - x_2 \right) \left(x_1 - x_2\right)} 
\end{align}


\subsubsection{$yyd$}

$3$ domain locations:
\begin{align}
  y_0 & = a_0+a_1 x_0+a_2 x_0^{2}  \\
   y_1 & = a_0+a_1 x_1+a_2 x_1^{2}  \\
   d_2 & = a_1+2 a_2 x_2  
\end{align}

 
\begin{align}
  a_0 & = \frac
{ \left( x_0 - 2 x_2 \right)  x_0 y_1 -  \left( x_1 - 2 x_2 \right)  x_1 y_0 -  \left( x_0 -  x_1 \right)  d_2 x_0 x_1}
{ \left( x_1 - 2 x_2+x_0 \right)   \left( x_0 - x_1  \right) } \\
   a_1 & = \frac
{ \left( x_0+x_1 \right)   \left( x_0 - x_1 \right)  d_2 - 2   \left( y_0 - y_1 \right)  x_2}
{ \left( x_1 - 2 x_2+x_0 \right)   \left( x_0 - x_1  \right) } \\
   a_2 & = \frac
{ -  \left( x_0 - x_1 \right)  d_2+y_0 - y_1}
{  \left( x_1 - 2 x_2+x_0 \right)   \left( x_0 - x_1 \right) } 
\end{align}


$2$ domain locations:
\begin{align}
  y_0 & = a_0+a_1 x_0+a_2 x_0^{2}  \\
   y_1 & = a_0+a_1 x_0+a_2 x_1^{2}  \\
   d_1 & = a_1+2 a_2 x_1  
\end{align}

 
\begin{align}
   a_0 & = \frac
{x_0^{2} y_1 - x_1^{2} y_0+2 \left(y_0 - y_1\right) x_0 x_1 - \left(x_0+x_1 \right) \left(x_0 - x_1\right) d_1 x_0}
{\left(x_0+x_1\right) \left( x_0 - x_1\right)} \\
   a_1 & = \frac
{\left(x_0+x_1\right) \left(x_0 - x_1\right) d_1 -  2 \left(y_0 - y_1\right) x_1}
{\left(x_0+x_1\right) \left(x_0 - x_1 \right)} \\
   a_2 & = \frac
{y_0 - y_1}
{\left(x_0+x_1\right) \left(x_0  - x_1\right)} 
\end{align}


\subsubsection{$ydd$}

$3$ domain locations:
\begin{align}
  y_0 & = \mu_0+\mu_1 x_0+\mu_2 x_0^{2}  \\
   d_1 & = \mu_1+2 \mu_2 x_1  \\
   d_2 & = \mu_1+2 \mu_2 x_2  
\end{align}

 
\begin{align}
  \mu_0 & = \frac
{ -  \left(  \left( x_0 - 2 x_2 \right)  d_1 x_0 - 2  \left( x_1 - x_2 \right)  y_0  \right) + \left( x_0 - 2 x_1 \right)  d_2 x_0}
{2 x_1 - 2 x_2} \\
   \mu_1 & = \frac
{ - d_1 x_2+d_2 x_1}
{x_1 - x_2} \\
   \mu_2 & = \frac
{d_1 - d_2}
{2 x_1 - 2 x_2} 
\end{align}


$2$ domain locations:
\begin{align}
  y_0 & = a_0+a_1 x_0+a_2 x_0^{2}  \\
   d_0 & = a_1+2 a_2 x_0  \\
   d_1 & = a_1+2 a_2 x_1  
\end{align}

 
\begin{align}
   a_0 & = \frac
{2 \left(x_0 - x_1\right) y_0 - d_1 x_0^{2} - \left(x_0 - 2 x_1\right) d_0 x_0 }
{2 \left(x_0 - x_1\right)} \\
   a_1 & = \frac
{ - \left(d_0 x_1 - d_1 x_0 \right)}
{x_0 - x_1} \\
   a_2 & = \frac
{d_0 - d_1}
{2 \left( x_0 - x_1\right)} 
\end{align}


\newgeometry{onecolumn=true}

\subsection{Cubic Monomial}

\subsubsection{$yyyy$}

\begin{align}
  y_0 & = a_0+a_1 x_0+a_2 x_0^{2}+a_3 x_0^{3}  \\
   y_1 & = a_0+a_1 x_1+a_2 x_1^{2}+a_3 x_1^{3}  \\
   y_2 & =  a_0+a_1 x_2+a_2 x_2^{2}+a_3 x_2^{3}  \\
   y_3 & = a_0+a_1 x_3+a_2 x_3^{2}+a_3 x_3^{3}  
\end{align}

 
\begin{align}
  a_0 & = \frac
{ 
\splitfrac{
\left(  
 \left( x_2 y_3 - x_3 y_2 \right)  x_1^{2} + 
 \left( x_2 - x_3 \right)  x_2 x_3  y_1 -  
 \left( x_2^{2} y_3 - x_3^{2} y_2 \right)  x_1 
\right)  x_0^{3} - 
\left( x_1 - x_2 \right)
\left( x_1 - x_3 \right)
\left( x_2 - x_3 \right) x_1 x_2 x_3 y_0 -  
}{
\left(  
 \left( x_2 y_3 - x_3  y_2 \right)  x_1^{3} + 
 \left( x_2+x_3 \right)   
 \left( x_2 - x_3 \right) x_2 x_3 y_1 - 
 \left( x_2 ^{3} y_3 - x_3^{3} y_2 \right)  x_1 
\right)  x_0^{2} + 
\left(  
 \left( x_2^{2} y_3 - x_3^{2}  y_2 \right)  x_1^{3} + 
 \left( x_2 - x_3 \right)  x_2^{2} x_3^{2} y_1 -  
 \left( x_2^{3} y_3 - x_3 ^{3} y_2 \right)  x_1^{2} 
\right)  x_0
  }}
{ \left( x_0 - x_1 \right)   \left( x_0  - x_2 \right)   
  \left( x_0 - x_3 \right)   \left( x_1 - x_2 \right)   
  \left( x_1 - x_3 \right)   \left( x_2 - x_3 \right) } \\
   a_1 & = \frac
{ -  \left(  \left(  \left( y_0 - y_2 \right)  x_3 ^{2} -  \left( y_0 - y_3 \right)  x_2^{2} \right)  x_1^{3} -  \left( x_2 - x_3 \right)   \left( y_0  - y_1 \right)  x_2^{2} x_3^{2} -  \left(  \left( y_0 - y_2 \right)  x_3^{3} -  \left( y_0 - y_3  \right)  x_2^{3} \right)  x_1^{2} \right) + \left(  \left( y_1 - y_3 \right)  x_2^{3} -   \left( y_2 - y_3 \right)  x_1^{3} -  \left( y_1 - y_2 \right)  x_3^{3} -  \left(  \left( y_1 - y_3  \right)  x_2^{2} -  \left( y_2 - y_3 \right)  x_1^{2} -  \left( y_1 - y_2 \right)  x_3^{2}  \right)  x_0 \right)  x_0^{2}}
{ \left( x_0 - x_1 \right)   \left( x_0 - x_2  \right)   \left( x_0 - x_3 \right)   \left( x_1 - x_2 \right)   \left( x_1 - x_3 \right)   \left(  x_2 - x_3 \right) } \\
   a_2 & = \frac
{ \left(  \left( y_0 - y_2 \right)  x_3 -  \left( y_0 - y_3  \right)  x_2 \right)  x_1^{3} -  \left( x_2+x_3 \right)   \left( x_2 - x_3 \right)   \left( y_0 -  y_1 \right)  x_2 x_3 -  \left(  \left( y_0 - y_2 \right)  x_3^{3} -  \left( y_0 - y_3 \right)  x_2^{3 } \right)  x_1 -  \left(  \left( y_1 - y_3 \right)  x_2^{3} -  \left( y_2 - y_3 \right)  x_1^{3} -   \left( y_1 - y_2 \right)  x_3^{3} -  \left(  \left( y_1 - y_3 \right)  x_2 -  \left( y_2 - y_3 \right)   x_1 -  \left( y_1 - y_2 \right)  x_3 \right)  x_0^{2} \right)  x_0}
{ \left( x_0 -  x_1 \right)   \left( x_0 - x_2 \right)   \left( x_0 - x_3 \right)   \left( x_1 - x_2 \right)    \left( x_1 - x_3 \right)   \left( x_2 - x_3 \right) } \\
   a_3 & = \frac
{ -  \left(  \left(   \left( y_0 - y_2 \right)  x_3 -  \left( y_0 - y_3 \right)  x_2 \right)  x_1^{2} -  \left( x_2 - x_3  \right)   \left( y_0 - y_1 \right)  x_2 x_3 -  \left(  \left( y_0 - y_2 \right)  x_3^{2} -  \left(  y_0 - y_3 \right)  x_2^{2} \right)  x_1 \right) + \left(  \left( y_1 - y_3 \right)  x_2^{2} -   \left( y_2 - y_3 \right)  x_1^{2} -  \left( y_1 - y_2 \right)  x_3^{2} -  \left(  \left( y_1 - y_3  \right)  x_2 -  \left( y_2 - y_3 \right)  x_1 -  \left( y_1 - y_2 \right)  x_3 \right)  x_0 \right)   x_0}
{ \left( x_0 - x_1 \right)   \left( x_0 - x_2 \right)   \left( x_0 - x_3 \right)    \left( x_1 - x_2 \right)   \left( x_1 - x_3 \right)   \left( x_2 - x_3 \right) } 
\end{align}


\subsubsection{$yyyd$}

$4$ domain locations:
\begin{align}
  y_0 & = \mu_0+\mu_1 x_0+\mu_2 x_0^{2}+\mu_3 x_0^{3}  \\
   y_1 & = \mu_0+\mu_1 x_1+\mu_2 x_1^{2}+\mu_3 x_1^{3}  \\
   y_2 & =  \mu_0+\mu_1 x_2+\mu_2 x_2^{2}+\mu_3 x_2^{3}  \\
   d_3 & = \mu_1+2 \mu_2 x_3+3 \mu_3 x_3^{2}  
\end{align}

 
\begin{align}
  \mu_0 & = \frac
{ \left(  \left(  \left( 2 x_2 - 3 x_3 \right)  x_3 -  \left( x_2 - 2 x_3 \right)  x_1  \right)  y_0+ \left( x_0 - x_1 \right)   \left( x_0 - x_2 \right)  d_3 x_0 \right)   \left( x_1 -  x_2 \right)  x_1 x_2 -  \left(  \left( x_1 - 2 x_3 \right)  x_1 y_2 -  \left( x_2 - 2 x_3 \right)  x_2  y_1 \right)  x_0^{3} -  \left(  \left( 2 x_1 - 3 x_3 \right)  x_1^{2} y_2 -  \left( 2 x_2 - 3 x_3  \right)  x_2^{2} y_1 \right)  x_0 x_3+ \left(  \left( x_1^{2} - 3 x_3^{2} \right)  x_1 y_2 -   \left( x_2^{2} - 3 x_3^{2} \right)  x_2 y_1 \right)  x_0^{2}}
{ \left(   \left( 2 x_2 - 3 x_3 \right)  x_3 -  \left( x_2 - 2 x_3 \right)  x_1 -  \left( x_2 - 2 x_3+x_1  \right)  x_0 \right)   \left( x_0 - x_1 \right)   \left( x_0 - x_2 \right)   \left( x_1 - x_2  \right) } \\
   \mu_1 & = \frac
{ -  \left(  \left( 2 x_1 - 3 x_3 \right)   \left( y_0 - y_2  \right)  x_1^{2} -  \left( 2 x_2 - 3 x_3 \right)   \left( y_0 - y_1 \right)  x_2^{2} -  \left( 2  x_0 - 3 x_3 \right)   \left( y_1 - y_2 \right)  x_0^{2} \right)  x_3 -  \left(  \left( x_1+x_2  \right)  x_0+x_1 x_2 \right)   \left( x_0 - x_1 \right)   \left( x_0 - x_2 \right)   \left( x_1 -  x_2 \right)  d_3}
{ \left(  \left( 2 x_2 - 3 x_3 \right)  x_3 -  \left( x_2 - 2 x_3  \right)  x_1 -  \left( x_2 - 2 x_3+x_1 \right)  x_0 \right)   \left( x_0 - x_1 \right)   \left( x_0  - x_2 \right)   \left( x_1 - x_2 \right) } \\
   \mu_2 & = \frac
{ \left( x_1^{2} - 3 x_3^{2}  \right)   \left( y_0 - y_2 \right)  x_1 -  \left( x_2^{2} - 3 x_3^{2} \right)   \left( y_0 - y_1  \right)  x_2 -  \left( x_0^{2} - 3 x_3^{2} \right)   \left( y_1 - y_2 \right)  x_0+ \left( x_1+ x_2+x_0 \right)   \left( x_0 - x_1 \right)   \left( x_0 - x_2 \right)   \left( x_1 - x_2 \right)  d_3 }
{ \left(  \left( 2 x_2 - 3 x_3 \right)  x_3 -  \left( x_2 - 2 x_3 \right)  x_1 -   \left( x_2 - 2 x_3+x_1 \right)  x_0 \right)   \left( x_0 - x_1 \right)   \left( x_0 - x_2 \right)    \left( x_1 - x_2 \right) } \\
   \mu_3 & = \frac
{ -  \left(  \left( x_1 - 2 x_3 \right)   \left(  y_0 - y_2 \right)  x_1 -  \left( x_2 - 2 x_3 \right)   \left( y_0 - y_1 \right)  x_2 -  \left( x_0 - 2 x_3  \right)   \left( y_1 - y_2 \right)  x_0 \right)  -  \left( x_0 - x_1 \right)   \left( x_0 - x_2  \right)   \left( x_1 - x_2 \right)  d_3}
{ \left(  \left( 2 x_2 - 3 x_3 \right)   x_3 -  \left( x_2 - 2 x_3 \right)  x_1 -  \left( x_2 - 2 x_3+x_1 \right)  x_0 \right)   \left( x_0 - x_1  \right)   \left( x_0 - x_2 \right)   \left( x_1 - x_2 \right) } 
\end{align}


$3$ domain locations:
\begin{align}
  y_0 & = \mu_0+\mu_1 x_0+\mu_2 x_0^{2}+\mu_3 x_0^{3}  \\
   y_1 & = \mu_0+\mu_1 x_1+\mu_2 x_1^{2}+\mu_3 x_1^{3}  \\
   y_2 & =  \mu_0+\mu_1 x_2+\mu_2 x_2^{2}+\mu_3 x_2^{3}  \\
   d_2 & = \mu_1+2 \mu_2 x_2+3 \mu_3 x_2^{2}  
\end{align}

 
\begin{align}
  \mu_0 & = \frac
{ \left( x_1^{2} y_2 - 2 x_1 x_2 y_2+x_2^{2} y_1 \right)  x_0^{3} -  \left( x_1 - x_2  \right) ^{2} x_1 x_2^{2} y_0 -  \left( x_1^{3} y_2 - 3 x_1 x_2^{2} y_2+2 x_2^{3} y_1  \right)  x_0^{2}+ \left( 2 x_1^{3} y_2 - 3 x_1^{2} x_2 y_2+x_2^{3} y_1 \right)  x_0 x_2 -   \left( x_0 - x_1 \right)   \left( x_0 - x_2 \right)   \left( x_1 - x_2 \right)  d_2 x_0 x_1 x_2 }
{ \left( x_0 - x_1 \right)   \left( x_0 - x_2 \right) ^{2}  \left( x_1 - x_2  \right) ^{2}} \\
   \mu_1 & = \frac
{ \left(  \left( 2 x_1 - 3 x_2 \right)   \left( y_0 - y_2  \right)  x_1^{2}+ \left( y_0 - y_1 \right)  x_2^{3} -  \left( 2 x_0 - 3 x_2 \right)   \left( y_1  - y_2 \right)  x_0^{2} \right)  x_2+ \left(  \left( x_1+x_2 \right)  x_0+x_1 x_2 \right)    \left( x_0 - x_1 \right)   \left( x_0 - x_2 \right)   \left( x_1 - x_2 \right)  d_2}
{ \left( x_0 - x_1 \right)   \left( x_0 - x_2 \right) ^{2}  \left( x_1 - x_2 \right) ^{2} } \\
   \mu_2 & = \frac
{ -  \left(  \left( x_1^{2} - 3 x_2^{2} \right)   \left( y_0 - y_2  \right)  x_1+2  \left( y_0 - y_1 \right)  x_2^{3} -  \left( x_0^{2} - 3 x_2^{2} \right)    \left( y_1 - y_2 \right)  x_0 \right)  -  \left( x_1+x_2+x_0 \right)   \left( x_0 - x_1 \right)    \left( x_0 - x_2 \right)   \left( x_1 - x_2 \right)  d_2}
{ \left( x_0 - x_1 \right)    \left( x_0 - x_2 \right) ^{2}  \left( x_1 - x_2 \right) ^{2}} \\
   \mu_3 & = \frac
{ \left(  x_1 - 2 x_2 \right)   \left( y_0 - y_2 \right)  x_1+ \left( y_0 - y_1 \right)  x_2^{2} -  \left( x_0 -  2 x_2 \right)   \left( y_1 - y_2 \right)  x_0+ \left( x_0 - x_1 \right)   \left( x_0 - x_2 \right)    \left( x_1 - x_2 \right)  d_2}
{ \left( x_0 - x_1 \right)   \left( x_0 - x_2 \right)  ^{2}  \left( x_1 - x_2 \right) ^{2}} 
\end{align}


\subsubsection{$yydd$}

Showing the most likely cases.

$4$ domain locations:
\begin{align}
  y_0 & = a_0+a_1 x_0+a_2 x_0^{2}+a_3 x_0^{3}  \\
   y_1 & = a_0+a_1 x_1+a_2 x_1^{2}+a_3 x_1^{3}  \\
   d_2 & =  a_1+2 a_2 x_2+3 a_3 x_2^{2}  \\
   d_3 & = a_1+2 a_2 x_3+3 a_3 x_3^{2}  
\end{align}

 
\begin{align}
   a_0 & = \frac
{\left(\left(2 \left(x_0^{2}+3 x_2 x_3\right) - 3 \left(x_2+x_3\right)  x_0\right) x_0 y_1 - \left(2 \left(x_1^{2}+3 x_2 x_3\right) - 3 \left(x_2+x_3\right)  x_1\right) x_1 y_0\right) \left(x_2 - x_3\right)+\left(x_0 x_1 - 2 x_0 x_2 - 2 x_1 x_2+3  x_2^{2}\right) \left(x_0 - x_1\right) d_3 x_0 x_1 - \left(x_0 x_1 - 2 x_0 x_3 - 2 x_1 x_3+3  x_3^{2}\right) \left(x_0 - x_1\right) d_2 x_0 x_1}
{\left(2 x_0^{2}+2  x_0 x_1 - 3 x_0 x_2 - 3 x_0 x_3+2 x_1^{2} - 3 x_1 x_2 - 3 x_1 x_3+6 x_2 x_3\right) \left(x_0 -  x_1\right) \left(x_2 - x_3\right)} \\
   a_1 & = \frac
{\left(\left(2 x_0^{2}+2 x_0  x_1 - 3 x_0 x_2+2 x_1^{2} - 3 x_1 x_2\right) \left(x_0 - x_1\right) d_3+6 \left(x_2 - x_3 \right) \left(y_0 - y_1\right) x_3\right) x_2 - \left(2 x_0^{2}+2 x_0 x_1 - 3 x_0 x_3+2  x_1^{2} - 3 x_1 x_3\right) \left(x_0 - x_1\right) d_2 x_3}
{\left(2 x_0 ^{2}+2 x_0 x_1 - 3 x_0 x_2 - 3 x_0 x_3+2 x_1^{2} - 3 x_1 x_2 - 3 x_1 x_3+6 x_2 x_3\right)  \left(x_0 - x_1\right) \left(x_2 - x_3\right)} \\
   a_2 & = \frac
{ - \left(\left(x_0 ^{2}+x_0 x_1+x_1^{2} - 3 x_2^{2}\right) \left(x_0 - x_1\right) d_3+3 \left(x_2+x_3 \right) \left(x_2 - x_3\right) \left(y_0 - y_1\right) - \left(x_0^{2}+x_0 x_1+x_1^{2} -  3 x_3^{2}\right) \left(x_0 - x_1\right) d_2\right)}
{\left(2 x_0^{2} +2 x_0 x_1 - 3 x_0 x_2 - 3 x_0 x_3+2 x_1^{2} - 3 x_1 x_2 - 3 x_1 x_3+6 x_2 x_3\right) \left( x_0 - x_1\right) \left(x_2 - x_3\right)} \\
   a_3 & = \frac
{\left(x_0+x_1 - 2 x_2 \right) \left(x_0 - x_1\right) d_3+2 \left(x_2 - x_3\right) \left(y_0 - y_1\right) -  \left(x_0+x_1 - 2 x_3\right) \left(x_0 - x_1\right) d_2}
{\left(2 x_0^{2 }+2 x_0 x_1 - 3 x_0 x_2 - 3 x_0 x_3+2 x_1^{2} - 3 x_1 x_2 - 3 x_1 x_3+6 x_2 x_3\right) \left( x_0 - x_1\right) \left(x_2 - x_3\right)} 
\end{align}


$2$ domain locations:
\begin{align}
  y_0 & = a_0+a_1 x_0+a_2 x_0^{2}+a_3 x_0^{3}  \\
   y_1 & = a_0+a_1 x_1+a_2 x_1^{2}+a_3 x_1^{3}  \\
   d_0 & =  a_1+2 a_2 x_0+3 a_3 x_0^{2}  \\
   d_1 & = a_1+2 a_2 x_1+3 a_3 x_1^{2}  
\end{align}

 
\begin{align}
  a_0 & =  \left(  -  \left( d_0 x_1+d_1 x_0 \right)   \left( x_0 - x_1 \right)  x_0 x_1+x_0^{3} y_1 - 3  x_0^{2} x_1 y_1+3 x_0 x_1^{2} y_0 - x_1^{3} y_0 \right)  /  \left( x_0 - x_1 \right) ^{3}  \\
   a_1  & =  \left(  \left(  \left( x_0+2 x_1 \right)   \left( x_0 - x_1 \right)  d_1 - 6  \left( y_0 - y_1  \right)  x_1 \right)  x_0+ \left( 2 x_0+x_1 \right)   \left( x_0 - x_1 \right)  d_0 x_1  \right)  /  \left( x_0 - x_1 \right) ^{3}  \\
   a_2 & =  \left(  -  \left(  \left( x_0+2 x_1 \right)    \left( x_0 - x_1 \right)  d_0 - 3  \left( x_0+x_1 \right)   \left( y_0 - y_1 \right)  \right)  -   \left( 2 x_0+x_1 \right)   \left( x_0 - x_1 \right)  d_1 \right)  /  \left( x_0 - x_1 \right) ^{3}   \\
   a_3 & =  \left(  \left( d_0+d_1 \right)   \left( x_0 - x_1 \right)  - 2  \left( y_0 - y_1 \right)   \right)  /  \left( x_0 - x_1 \right) ^{3}  
\end{align}


\subsubsection{$yddd$}
$4$ domain locations:

\begin{align}
  y_0 & = \mu_0+\mu_1 x_0+\mu_2 x_0^{2}+\mu_3 x_0^{3}  \\
   d_1 & = \mu_1+2 \mu_2 x_1+3 \mu_3 x_1^{2}  \\
   d_2 & = \mu_1+2 \mu_2  x_2+3 \mu_3 x_2^{2}  \\
   d_0 & = \mu_1+2 \mu_2 x_0+3 \mu_3 x_0^{2}  
\end{align}

 
\begin{align}
  \mu_0 & = \frac
{ \left(  \left( x_0 - 3 x_1 \right)  d_2 x_0^{2}+6  \left( x_0 - x_2 \right)    \left( x_1 - x_2 \right)  y_0 \right)   \left( x_0 - x_1 \right)  -  \left( x_0 - x_2 \right)    \left( x_0 - 3 x_2 \right)  d_1 x_0^{2} -  \left( 2  \left( x_0^{2}+3 x_1 x_2 \right)  - 3   \left( x_1+x_2 \right)  x_0 \right)   \left( x_1 - x_2 \right)  d_0 x_0}
{6   \left( x_0 - x_1 \right)   \left( x_0 - x_2 \right)   \left( x_1 - x_2 \right) } \\
   \mu_1 & = \frac
{ \left(  \left( x_0 - x_1 \right)  d_2 x_1 -  \left( x_0 - x_2 \right)  d_1 x_2 \right)  x_0+  \left( x_1 - x_2 \right)  d_0 x_1 x_2}
{ \left( x_0 - x_1 \right)   \left( x_0 - x_2  \right)   \left( x_1 - x_2 \right) } \\
   \mu_2 & = \frac
{ \left( x_0+x_2 \right)   \left( x_0  - x_2 \right)  d_1 -  \left( x_1+x_2 \right)   \left( x_1 - x_2 \right)  d_0 -  \left( x_0+x_1  \right)   \left( x_0 - x_1 \right)  d_2}
{2  \left( x_0 - x_1 \right)   \left( x_0  - x_2 \right)   \left( x_1 - x_2 \right) } \\
   \mu_3 & = \frac
{ -  \left(  \left( x_0 - x_2  \right)  d_1 -  \left( x_1 - x_2 \right)  d_0 \right) + \left( x_0 - x_1 \right)  d_2}
{3  \left( x_0 - x_1 \right)   \left( x_0 - x_2 \right)   \left( x_1 - x_2 \right) } 
\end{align}


\restoregeometry
  

%\section{Lagrange basis}
